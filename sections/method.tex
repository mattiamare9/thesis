\section{Evaluation Metric}

To quantitatively assess the accuracy of the reconstructed sound field, we
employ the \emph{Normalized Mean Squared Error} (NMSE), a widely used metric
in acoustic field estimation, spatial audio, and inverse problems.  NMSE
measures the relative discrepancy between the reconstructed complex pressure
and the ground-truth reference field.

\subsection{Definition}

Let $u_{\mathrm{ref}}(x)$ denote the reference sound field and
$\hat u(x)$ the estimated field, both evaluated at a set of validation
positions $\{x_m^{(v)}\}_{m=1}^{M_v}$.  
The NMSE is defined as
%
\begin{equation}
    \mathrm{NMSE}
    =
    \frac{
        \sum_{m=1}^{M_v}
        \bigl|
            u_{\mathrm{ref}}(x_m^{(v)})
            -
            \hat u(x_m^{(v)})
        \bigr|^2
    }{
        \sum_{m=1}^{M_v}
        \bigl|
            u_{\mathrm{ref}}(x_m^{(v)})
        \bigr|^2
        + \epsilon
    },
    \label{eq:nmse}
\end{equation}
%
where $\epsilon>0$ is a small constant included only to avoid division by zero
in degenerate cases.  
The numerator corresponds to the reconstruction error, while the denominator
normalizes the value by the total energy of the true sound field.

For reporting convenience, NMSE values are often expressed in decibels:
%
\begin{equation}
    \mathrm{NMSE}_{\mathrm{dB}}
    = 10 \log_{10}(\mathrm{NMSE}).
    \label{eq:nmse-db}
\end{equation}

\subsection{Why NMSE for Acoustic Pressure Fields?}

NMSE is particularly well suited for evaluating complex acoustic pressure
fields for several reasons:

\paragraph{1) Scale invariance.}
The denominator normalizes by the energy of the reference field.  
This ensures that the metric is unaffected by the absolute amplitude of the
sound pressure, which may vary with frequency, microphone configuration, or
source–receiver distance.  
Thus, NMSE allows fair comparison across frequencies and across distinct
kernels.

\paragraph{2) Physically meaningful for complex-valued fields.}
Acoustic pressures are complex quantities, encoding both magnitude and phase.
The squared modulus
\(
|u_{\mathrm{ref}}(x) - \hat u(x)|^2
\)
captures phase errors and amplitude errors simultaneously, and therefore
quantifies the discrepancy in the physically relevant quantity.

\paragraph{3) Robustness across heterogeneous validation geometries.}
In many sound-field estimation tasks, validation microphones are placed in
regions with different energy levels (e.g.\ inside vs.\ outside measurement
surfaces).  
By normalizing by $\|u_{\mathrm{ref}}\|^2$, NMSE remains comparable across
evaluation regions with different field magnitudes.

\medskip

Moreover, NMSE has been adopted extensively in the spatial acoustics literature,
including spherical harmonics interpolation, Herglotz-based reconstructions,
plane-wave decomposition methods, and prior kernel-based approaches.  
Using NMSE thus allows direct comparison with related methods.

\subsection{Interpretation}

An NMSE of $0$ corresponds to perfect reconstruction, while larger values
indicate poorer accuracy.  
Expressed in decibels, negative NMSE values (e.g.\ $-15$~dB) indicate that the
error power is much smaller than the signal power, whereas values close to
0~dB suggest errors comparable to the signal itself.

Since NMSE aggregates errors over all validation points, it provides a global
measure of reconstruction quality, sensitive to global shape accuracy of the wavefront.

For these reasons, NMSE serves as a principled and physically meaningful
metric for comparing the uniform, directed, neural, and hybrid kernels
investigated in this work.


