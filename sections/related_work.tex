\section{Related Work}

Research on sound-field reconstruction from sparse measurements has produced a
variety of models and kernel constructions, many of which exploit the
structure of solutions to the Helmholtz equation or are based on physically
motivated parametrizations.

\subsection{Plane-Wave Representations}

A classical representation of Helmholtz solutions is the plane-wave (Herglotz)
expansion
%
\begin{equation}
    u(x)
    =
    \int_{S^2} a(\eta)\, e^{ik\,\eta \cdot x}\, d\eta ,
    \label{eq:rw-herglotz}
\end{equation}
%
where $a(\eta)$ is a directional amplitude function.
Many spatial audio and field-interpolation methods approximate
\eqref{eq:rw-herglotz} using a finite set of directions
$\{\eta_d\}_{d=1}^N$ and quadrature weights $\{\sigma_d\}_{d=1}^N$, leading to
%
\begin{equation}
    u(x)
    \approx
    \sum_{d=1}^N \sigma_d\, e^{ik\,\eta_d \cdot x}.
    \label{eq:rw-discretepw}
\end{equation}
%
Such discrete plane-wave models form the basis of uniform kernels, neural
plane-wave kernels, and residual models used in data-driven acoustics.

\subsection{Isotropic and Bessel-Type Kernels}

Integrating plane waves uniformly over the sphere yields an isotropic kernel
that depends only on the distance $\|x-x'\|$:
%
\begin{equation}
    \int_{S^2} e^{ik\,\eta\cdot(x-x')} d\eta
    = 4\pi\, j_0(k\|x-x'\|),
    \label{eq:rw-bessel}
\end{equation}
%
where $j_0$ is the spherical Bessel function of order zero,
%
\[
    j_0(t) = \frac{\sin t}{t}.
\]
%
Related formulations employ the Bessel function of the first kind $J_0$, and
in anisotropic settings the modified Bessel function $I_0$.
These kernels are widely used in wave-field synthesis, near-field holography,
and spatial filtering, as they provide rotationally symmetric reproducing
kernels consistent with the Helmholtz equation.

\subsection{Directional and Exponential Kernels}

To extend isotropic models and represent anisotropic propagation, several
works introduce \emph{directionally biased} kernels obtained by modifying the
uniform (isotropic) plane-wave distribution with an exponential weight.  
Starting from the uniform weight $w(\eta)=1$, one introduces the directional
emphasis
%
\begin{equation}
    w(\eta)
    \;\propto\;
    \exp\!\bigl(\beta\, \eta \cdot d\bigr),
    \label{eq:rw-expweight}
\end{equation}
%
where $d \in S^2$ is a preferred direction and $\beta>0$ controls how sharply
the weight concentrates around $d$:
%
\begin{itemize}
    \item when $\beta = 0$, $w(\eta)=1$ and the kernel reduces to the uniform
          isotropic model;
    \item as $\beta$ increases, directions $\eta$ aligned with $d$ are
          exponentially amplified, while directions opposite to $d$ are
          attenuated.
\end{itemize}

Thus, \eqref{eq:rw-expweight} acts as a tunable ``directional spotlight’’ on
the sphere of propagation directions.  
This idea naturally fits the physics of point-source radiation, where the
wavefield exhibits stronger coherence along outward-propagating directions.

When this weighted distribution is substituted into the Herglotz
representation, the resulting kernel becomes a \emph{directionally biased}
variant of the isotropic Bessel-type kernel.  
A representative form arising in the literature is
%
\[
    \kappa(x,x')
    =
    J_0\!\left(
        \bigl\| k(x-x') - i\beta\,d \bigr\|
    \right),
\]
%
which can be interpreted as a complex-shifted Bessel function whose argument
is displaced along the preferred direction $d$.  
This complex displacement encodes the exponential weighting
\eqref{eq:rw-expweight} and produces a kernel whose correlation is strongest
along $d$ and decreases for directions orthogonal or opposite to it.

Directional exponential kernels of this form have been widely used in
sound-field control, directional beamforming, and propagation modeling in both
free-field and room-acoustic scenarios, as they offer a principled mechanism
for emphasizing a specific propagation direction while retaining the
Helmholtz-compliant structure of Bessel-based kernels.


\subsection{Learning-Based and Adaptive Kernels}

Recent approaches propose to learn directional weights directly from data,
often by modeling $a(\eta)$ or $w(\eta)$ with a neural network.
In these methods, the kernel takes the form
%
\[
    \kappa(x,x') =
    \sum_{d=1}^N
        \sigma_d\, W(k\eta_d)
        e^{ik\,\eta_d\cdot(x-x')},
\]
where $W$ is a trainable positive function (typically an MLP).
This strategy enables data-driven adaptation of directional patterns while
preserving the Helmholtz structure of the kernel, and has shown promising
performance for acoustic field reconstruction, especially in sparse or
irregular microphone configurations.

\subsection{Relation to the Present Work}

The approaches above motivate the kernel families evaluated in this paper:
%
\begin{itemize}
    \item \emph{uniform kernels} based on isotropic spherical Bessel functions,
    \item \emph{directed kernels} obtained from exponential directional weights,
    \item \emph{neural plane-wave kernels} that learn anisotropic weights 
          from data,
    \item and \emph{hybrid (directed + neural) kernels} combining strong
          analytic directionality with learned residual components.
\end{itemize}
%
All these kernels remain consistent with the Helmholtz equation through their
plane-wave or Bessel-based construction, and their comparative performance in
sound-field estimation is the primary focus of this study.
