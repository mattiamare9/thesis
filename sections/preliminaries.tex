\section{Problem Statement and Preliminaries}
This section introduces the notation, geometric assumptions, and mathematical
background used throughout the kernel formulations.

\bigskip

We consider a three-dimensional region $\Omega \subset \mathbb{R}^3$ in which the acoustic field is assumed to be
\emph{source-free}. Suppose also that there is an unknown number of
sources outside of $\Omega$ generating sound waves that combine into a pressure distribution denoted as $u(\vect{x},k)$
For a fixed angular frequency $\omega = 2\pi f$, the complex acoustic pressure field 
$u(\vect{x},k) \in \mathbb{C}$ satisfies the homogeneous Helmholtz equation
%
\begin{equation}
    \nabla^2 u(\vect{x},k) + k^2 u(\vect{x},k) = 0,
    \qquad \forall x \in \Omega,
    \label{eq:helmholtz}
\end{equation}
%
where $k = \omega / c$ is the wavenumber and $c$ is the speed of sound.

Our objective is to estimate the sound field $u(\vect{x},k)$ at any position $x \in \Omega$ from a \emph{snapshot}
consisting of $M$ microphone measurements
\[
    \{s_m(k)\}_{m=1}^M, \qquad s_m(k) = u(\vect{x}_m,k),
\]
measured at known microphone positions $\{\vect{x}_m\}_{m=1}^M \subset \Omega$.
An illustration of this problem setting is given in Fig.~1 of the original article.

For notational simplicity, we omit $k$ from the arguments whenever no ambiguity arises.

\subsection{Estimation Objective}

Following the standard formulation in kernel-based sound field estimation, we seek an estimate
$u \colon \Omega \to \mathbb{C}$ that minimizes the regularized empirical loss
%
\begin{equation}
    L(u)
    :=
    \sum_{m=1}^M |s_m - u(\vect{x}_m)|^2
    \;+\;
    \lambda \lVert u \rVert_H^2,
    \label{eq:loss}
\end{equation}
%
where $\lambda > 0$ is a regularization constant, and $\|\cdot\|_H$ is the norm associated with a
functional space $H$ to be defined.
The estimator is therefore the unique minimizer
%
\begin{equation}
    \hat{u}
    =
    \arg\min_{u \in H} L(u).
    \label{eq:krr-solution}
\end{equation}

The choice of the space $H$ is crucial: it must encode the physical constraint
\eqref{eq:helmholtz}, possess a well-defined norm, and allow tractable computation of
$\hat{u}$. Moreover, we can guarantee a unique solution with a closed form if $H$ is
a \emph{reproducing kernel Hilbert space} (RKHS).

\subsection{Representation of Helmholtz Solutions}

To ensure physical admissibility, the authors construct $H$ using the classical representation of 
Helmholtz solutions as \emph{Herglotz wave functions}.
Any solution $u$ of \eqref{eq:helmholtz} in a source-free region can be written as
%
\begin{equation}
    u(\vect{x})
    =
    T(\tilde{u})
    :=
    \int_{S^2} e^{ik \eta \cdot \vec{x}} \tilde{u}(\eta) \, d\eta,
    \label{eq:herglotz}
\end{equation}
%
where $S^2$ denotes the unit sphere, $\eta \in S^2$ is a directional variable, and $\tilde{u}$ is a square-integrable
function over the sphere (the \emph{directional representation} of $u$).
This integral representation ensures that $u$ always satisfies the Helmholtz equation, since
$(\nabla^2 + k^2)e^{ik\eta \cdot \vec{x}} = 0$ for every $\eta$.

The authors introduce a \emph{weight function} $w \colon S^2 \to \mathbb{R}_{>0}$ and require
%
\begin{equation}
    \int_{S^2}
        \frac{|\tilde{u}(\eta)|^2}{w(\eta)} \, d\eta < \infty.
    \label{eq:l2w}
\end{equation}
%
In this case we write $\tilde{u} \in L^2(S^2, w)$.
The weight $w$ plays an essential role in shaping the directional characteristics of the field
and will later become the object to be \emph{adapted} from data.

Using the mapping $T$ in \eqref{eq:herglotz}, the functional space is defined as
%
\begin{equation}
    H
    :=
    \left\{
        u = T(\tilde{u})
        \;\big|\;
        \tilde{u} \in L^2(S^2, w)
    \right\}.
    \label{eq:H-def}
\end{equation}

\subsection{Inner Product and Structure of the Space}

Since $L^2(S^2, w)$ is a Hilbert space, the induced inner product on $H$ is
%
\begin{equation}
    \langle u_1, u_2 \rangle_H
    =
    4\pi
    \int_{S^2}
        \frac{\tilde{u}_1(\eta)^{\ast}\tilde{u}_2(\eta)}
             {w(\eta)}
    \, d\eta.
    \label{eq:inner-product}
\end{equation}
%
The completeness of $H$ under this inner product follows from the proportionality between the
norms of $T(\tilde{u})$ and $\tilde{u}$, proven in Appendix~B of the original paper.

Importantly, for any continuous weight function $w$, the space $H$ is guaranteed to be a
reproducing kernel Hilbert space (RKHS), which implies:
1) the minimizer \eqref{eq:krr-solution} exists and is unique, and  
2) the solution can be written via a kernel of the form $\kappa(\cdot, \vect{x}_m)$.

The RKHS machinery itself is not detailed here, but its consequence is the kernel ridge regression
form of the solution.



\subsection{Kernel Ridge Regression Form}

Once the functional space $H$ is chosen as in the previous subsection, the
minimizer of \eqref{eq:loss} must belong to $H$ and must satisfy the
Helmholtz equation by construction.  
To guarantee the existence of a unique minimizer of \eqref{eq:krr-solution} and
to obtain it in closed form, we require $H$ to be a reproducing kernel Hilbert
space (RKHS).  
This means that there exists a \emph{reproducing kernel}
\(\kappa : \Omega \times \Omega \to \mathbb{C}\) such that
%
\begin{equation}
    \langle \kappa(\cdot,\vect{x}), u \rangle_H = u(\vect{x}),
    \qquad \forall\, \vect{x}\in\Omega,\ \forall\,u\in H.
    \label{eq:rk-property}
\end{equation}

Under this assumption, the classical representer theorem ensures that the solution of
\eqref{eq:krr-solution} is a finite linear combination of kernel sections
evaluated at the microphone positions.  
Hence, the estimated sound field $\hat u$ can be written as
%
\begin{equation}
    \hat{u}(\vect{x})
    =
    \sum_{m=1}^M \alpha_m\, \kappa(\vect{x},\vect{x}_m)
    =
    \kappa(\vect{x})\,\vect{\alpha},
    \label{eq:krr-estimator}
\end{equation}
%
where the vector of coefficients $\vect{\alpha} = [\alpha_1,\dots,\alpha_M]^\top$
depends solely on the microphone measurements.  Here we use the shorthand
%
\[
    \kappa(\vect{x})
    :=
    \bigl[\kappa(\vect{x},\vect{x}_1),\dots,\kappa(\vect{x},\vect{x}_M)\bigr].
\]

Substituting this form into the loss function \eqref{eq:loss} yields the
standard kernel ridge regression (KRR) linear system.  
We define the Gram matrix
%
\begin{equation}
    \vect{K} \in \mathbb{C}^{M \times M}, \qquad
    K_{mn} = \kappa(\vect{x}_m, \vect{x}_n),
    \label{eq:gram-matrix}
\end{equation}
%
and the measurement vector
%
\[
    s = [s_1,\dots,s_M]^\top.
\]

Then the coefficients $\vect{\alpha}$ minimizing~\eqref{eq:loss} are given by
%
\begin{equation}
    \vect{\alpha}
    =
    (\vect{K} + \lambda I)^{-1} s.
    \label{eq:krr-solution-alpha}
\end{equation}

Combining \eqref{eq:krr-estimator} and \eqref{eq:krr-solution-alpha}, the
resulting KRR estimate is
%
\begin{equation}
    \hat{u}(\vect{x})
    =
    \kappa(\vect{x})\,(\vect{K}+\lambda I)^{-1}s.
    \label{eq:krr-final}
\end{equation}

Physically, the kernel $\kappa$ encodes the directional and structural
properties of all admissible sound fields in the space $H$, while the Gram
matrix $\vect{K}$ expresses the pairwise similarity of microphone observations under
that model.  
Thus, once $\kappa$ is chosen (i.e., equivalently, once the weight function $w$
determining the kernel is set) sound-field reconstruction reduces to solving a
single linear system \eqref{eq:krr-solution-alpha}.  
In the remainder of this manuscript, the focus is on designing and comparing
different physically motivated kernels $\kappa$ and investigating how well
each one models the spatial structure of the true sound field.

